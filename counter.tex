\section{Monotone Counter}
\label{section:monotone-counter}

\remark{{\footnotesize Coq code: \url{https://github.com/abeln/iris-practice/blob/master/counter.v}}}
\\

This is the counter example from Section 7.7 of the notes. The module exports three methods:

\begin{itemize}

\item \texttt{newCounter} creates a new counter. Counters are represented simply with an int reference.

\begin{verbatim}
Definition newCounter : val := \lambda: <>, ref #0.
\end{verbatim}

\item \texttt{read} returns the value currently stored in a counter.

\begin{verbatim}
Definition read : val := \lambda: "c", !"c".
\end{verbatim}

\item \texttt{incr} takes a counter, increments it, and returns unit.

\begin{verbatim}
Definition incr : val :=
  rec: "incr" "c" :=
    let: "n" := !"c" in
    let: "m" := #1 + "n" in
    if: CAS "c" "n" "m" then #() else "incr" "c".
\end{verbatim}

\end{itemize}

The client for the counter is a program that instantiates a counter, spawns two threads each incrementing the counter, and then reads the value off the counter:

\begin{verbatim}
  Definition client : val :=
   \lambda: <>,
     let: "c" := newCounter #() in
     ((incr "c") ||| (incr "c")) ;;
     read "c".   
\end{verbatim}

\subsection{Specs}

We can prove two different specs for the code above:
\begin{itemize}
\item Authoritative RA

In the first spec, our resource algebra is $\authra{\natra}$. Elements of this RA are of the form either $\aelem{n}$ (authoritative) or $\naelem{n}$ (non-authoritative), where $n \in \natra$.

The important properties of this RA (for the counter example) are:
\begin{itemize}
\item $\validelem{\prodelem{\aelem{0}}{\naelem{0}}}$
\item $\validelem{\prodelem{\aelem{m}}{\naelem{n}}}$ implies $m \ge n$
\item $\fpupd{\prodelem{\aelem{m}}{\naelem{n}}}{\prodelem{\aelem{(m + 1)}}{\naelem{(n + 1)}}}$
\end{itemize}

Then we can define our predicate for the counter:
\[ \isCounter{l}{n}{\gamma} = \ownr{\naelem{n}} {\gamma} * \exists s.\inv{ $\exists m. \pointsto{l}{m} \land \ownr{\aelem{m}}{\gamma} $ }{s} \]

\item Authoritative RA + Fractions
\end{itemize}