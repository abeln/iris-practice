\section{Monotone Counter}
\label{section:monotone-counter}

\remark{{\footnotesize Coq code: \url{https://github.com/abeln/iris-practice/blob/master/counter.v}}}
\\

This is the counter example from Section 7.7 of the notes. The module exports three methods:

\begin{itemize}

\item \texttt{newCounter} creates a new counter. Counters are represented simply with an int reference.

\begin{verbatim}
Definition newCounter : val := \lambda: <>, ref #0.
\end{verbatim}

\item \texttt{read} returns the value currently stored in a counter.

\begin{verbatim}
Definition read : val := \lambda: "c", !"c".
\end{verbatim}

\item \texttt{incr} takes a counter, increments it, and returns unit.

\begin{verbatim}
Definition incr : val :=
  rec: "incr" "c" :=
    let: "n" := !"c" in
    let: "m" := #1 + "n" in
    if: CAS "c" "n" "m" then #() else "incr" "c".
\end{verbatim}

\end{itemize}

The client for the counter is a program that instantiates a counter, spawns two threads each incrementing the counter, and then reads the value off the counter:

\begin{verbatim}
  Definition client : val :=
   \lambda: <>,
     let: "c" := newCounter #() in
     ((incr "c") ||| (incr "c")) ;;
     read "c".   
\end{verbatim}

\subsection{Specs}

We can prove two different specs for the code above:
\begin{itemize}
\item Authoritative RA

In the first spec, our resource algebra is $\authra{\natra}$. Elements of this RA are of the form either $\aelem{n}$ (authoritative) or $\naelem{n}$ (non-authoritative), where $n \in \natra$.

The important properties of this RA (for the counter example) are:
\begin{itemize}
\item $\validelem{\prodelem{\aelem{0}}{\naelem{0}}}$
\item $\validelem{\prodelem{\aelem{m}}{\naelem{n}}}$ implies $m \ge n$
\item $\fpupd{\prodelem{\aelem{m}}{\naelem{n}}}{\prodelem{\aelem{(m + 1)}}{\naelem{(n + 1)}}}$
\item $|\naelem{n}| = \naelem{n}$
\end{itemize}

Then we can define our predicate for the counter:
\[ \isCounter{l}{n}{\gamma} = \ownr{\naelem{n}} {\gamma} * \exists s.\inv{ $\exists m. \pointsto{l}{m} \land \ownr{\aelem{m}}{\gamma} $ }{s} \]

This predicate is persistent because the invariant component is persistent and $\naelem{n}$ is also persistent (via the \textsc{Persistently-core} rule), because the core of $\naelem{n}$ is $\naelem{n}$ itself .

The intuition behind the predicate is that if we own $\isCounter{l}{n}{\gamma}$ then, at an atomic step (e.g. a load), we can know that $l$ is allocated and it points to a value that's $\ge n$.

Additionally, after an increment, we can update the resource to $\isCounter{l}{n + 1}{\gamma}$.

We then prove the following specs for the counter methods:
\begin{itemize}
\item \hot{$\top$}{$\text{newCounter ()}$}{$l. \exists \gamma. \isCounter{l}{0}{\gamma}$}
\item \hot{$\isCounter{l}{n}{\gamma}$}{$\app{\text{read}}{l}$}{$v. v \ge n$}. Notice how we don't have to give the counter back, because it's persistent.
\item \hot{$\isCounter{l}{n}{\gamma}$}{$\app{\text{incr}}{l}$}{$\isCounter{l}{n + 1}{\gamma}$}
\end{itemize}

We can also prove the client spec:

\[ \hot{\top}{\app{\text{client}}{()}}{n. n \ge 1} \]

The ``problem'' with this specification $isCounter$ doesn't tell us anything about what other threads are doing with the counter. So in the verification of the client code, after both threads return, we know 

\[ \isCounter{l}{1}{\gamma}  * \isCounter{l}{1}{\gamma} \equiv \isCounter{l}{1}{\gamma} \]

So we know that the counter must be $\ge 1$, but we don't know that it's 2.

\item Authoritative RA + Fractions

The second spec we can give is more precise, and allows us to conclude that the client reads exactly $2$ once the two threads finish.

The resource algebra we use is $\authra{\optra{\fracra \times \natra}}$:
\begin{itemize}
\item $\fracra$ is the RA of positive fractions, where the valid elements are those $\le 1$.
\item The $?$ is the \emph{optional RA} construction. In this case, it's needed because to use the authoritative RA we need the argument algebra to be \emph{unital}. And $\fracra \times \natra$ is not unital because $0$ is not an element of $\fracra$.
\end{itemize}

The important properties of this RA are:
\begin{itemize}
\item $\validelem{\prodelem{\naelem{(1, 0)}}{\aelem{(1, 0)}}}$
\item $\validelem{\prodelem{\naelem{(p, n)}}{\aelem{(1, m)}}}$ implies $m \ge n$
\item $\validelem{\prodelem{\naelem{(1, n)}}{\aelem{(1, m)}}}$ implies $m = n$
\item $\fpupd{\prodelem{\naelem{(p, n)}}{\aelem{(q, m)}}}{\prodelem{\naelem{(p, n + 1)}}{\aelem{(q, m + 1)}}}$
\end{itemize}

With this RA, we get the new counter resource

\[ \isCounterF{l}{n}{p}{\gamma} = \ownr{\naelem{(p, n)}} {\gamma} * \exists s.\inv{ $\exists m. \pointsto{l}{m} \land \ownr{\aelem{(1, m)}}{\gamma} $ }{s} \]

\end{itemize}

This resource is no longer persistent, because the core $|p|$ is undefined. However, we can show the following

\[ \isCounterF{l}{n + m}{p + q}{\gamma} \dashv \vdash  \isCounterF{l}{n}{p}{\gamma} *  \isCounterF{l}{m}{q}{\gamma}    \] 
