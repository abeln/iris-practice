\section{Locks and Coarse-Grained Bags}
\label{section:locks-and-coarse-graned-bags}

\remark{{\footnotesize Coq code: \url{https://github.com/abeln/iris-practice/blob/master/lock.v}}}

This is the spin lock example from Section 7.6 of the notes.

The spin lock is a module with three methods:
\begin{itemize}

\item \texttt{newlock} creates a new lock. Locks are represented as a reference to a boolean. If \texttt{false}, the lock is free; if \texttt{true}, then it's taken.

\item \texttt{acquire} uses a CAS cycle to spin until it can acquire the lock.

\item  \texttt{release} just sets the lock to \texttt{false}.

\end{itemize}

The code is as above:

\begin{verbatim}
  Definition newlock : val := \lam: <>, ref #false.
    
  Definition acquire : val :=
    rec: "acquire" "l" :=
      if: CAS "l" #false #true then #() else "acquire" "l".
    
  Definition release : val := \lam: "l", "l" <- #false.
\end{verbatim}

\subsection{Spec}

The involved RA is $\exclra{\unittyp}$.
\begin{itemize}
\item Notice that unlike other RA constructions, $\exclra{S}$ just requires $S$ to be a \emph{set}, as opposed to another RA.

\item The exclusive RA is defined by adding an element $\bot$ to the carrier set. For any two elements $x, y \ne \bot$, we have $\prodelem{x}{y} = \bot$. Every element except for $\bot$ is valid.

\item What this means is that if we own $\ownr{x}{\gamma}$, no other thread can own another $\ownr{y}{\gamma}$, because their product would be invalid.
\end{itemize}

The lock predicate is then

\[ \isLock{l}{P}{\gamma} = \exists{s}{\inv{$\pointsto{l}{\falselit} \land P \land \ownr{\unitlit}{\gamma} \lor \pointsto{l}{\truelit}  $}{s}} \]

A first observation is that the lock protects an \emph{arbitrary predicate} $P$, which is what the client uses to prove its correctness.

The intuition is the following:
\begin{itemize}

\item If the lock is \emph{unlocked} ($\pointsto{l}{\falselit}$), then we need to \emph{give away} to the invariant the predicate $P$ protected by the lock \emph{and} the key $\unitlit$. Because $\unitlit \in \exclra{\unittyp}$, we \emph{know that no other key can be created}.

\item When the lock is \emph{locked} ($\pointsto{l}{\truelit}$), we don't need to give away any resources.

\item When the lock transitions from unlocked to locked, we \emph{gain} resources.

\item When the lock transitions from locked to unlocked, we \emph{lose} resources.

\end{itemize}

The last two points are reflected in the specs for the lock methods:

\begin{itemize}

\item $\hot{P}{\texttt{newlock } ()}{l.  \exists \gamma. \isLock{l}{P}{\gamma}}$.

Notice that even though only one element of $\exclra{\unittyp}$ is allowed, this restriction is \emph{per location/ghost name}, so we're allowed to allocate a new (exclusive) ghost resource, proviced we furnish a new ghost name $\gamma$.

\item $\hot{\isLock{l}{P}{\gamma}}{\texttt{acquire } l}{P * \ownr{\unitlit}{\gamma}}$

After acquiring the lock, we gain ownership of the predicate $P$ and the key $\unitlit$.

\item $\hot{\isLock{l}{P}{\gamma} * P * \ownr{\unitlit}{\gamma}}{\texttt{release } l}{\top}$

To release the lock, we need to show that $P$ continues to hold, and that we have the key $\ownr{\unitlit}{\gamma}$.

\end{itemize}

Also note that the lock predicate is \emph{persistent}, since it's protected inside an invariant. This is important because multiple threads need to know that a given lock exists (so they can synchronize).

\subsection{Client: Coarsed-Grained Bags}

This client is also from the notes, and it consists of a \emph{bag} data structure. We can create new bags, add items to it, and remove (the first) items from it.

\remark{The implementation of the bag looks just like a stack. But the specification says nothing about the order of insertions/removals, that's why it's a bag and not a stack}.

The bag code follows:

\begin{verbatim}
Section bag_code.

  (* A bag is a pair of (optional list of values, lock)  *)
  Definition new_bag : val :=
    \lam: <>, (newlock #(), ref NONEV).

  (* Insert a value into the bag. Return unit. *)
  Definition insert : val :=
    \lam: "bag" "v",
      let: "lst" := Snd "bag" in
      let: "lock" := Fst "bag" in
      acquire "lock";;
      "lst" <- SOME ("v", !"lst");;
      release "lock";;        
      #().        

  (* Remove the value last-added to the bag (wrapped in an option), if the
     bag is non-empty. If the bag is empty, return NONE. *)
  Definition remove : val :=
    \lam: "bag",
      let: "lst" := Snd "bag" in
      let: "lock" := Fst "bag" in
      acquire "lock";;
      let: "res" :=
         match: !"lst" with
           NONE => NONEV
         | SOME "pair" =>
           "lst" <- Snd "pair";;
           SOME (Fst "pair")
         end
      in           
      release "lock";;
      "res".        
  
End bag_code.
\end{verbatim}

\subsubsection{Bag Specification}

The spec we want to show says only elements satisfying some predicate $\Phi$ can be added to the bag. In turn, we guarantee that any element $x$ removed from the bag satisfies $\Phi x$.

The bag predicate looks like

\[ \isBag{b}{\Phi}{\gamma}  = \exists l, lst. b = (lock, lst) \land \isLock{l}{\exists n, vs. \pointsto{lst}{vs} \land \baglist{vs}{n}{\Phi}}{\gamma}  \]

In turn, $\baglist{vs}{n}{\Phi}$ is a predicate that expresses that $vs$ is a HeapLang list of $n$ elements, all of which satisfy $\Phi$.

\begin{align*}
\baglist{vs}{n}{\Phi} =& (n = 0 \land vs = \nonelit) \\
& \lor (n = S n' \land \\
& \exists v, vs'. vs = \somelit{(v, vs')} \land \Phi v \land \baglist{vs}{n'}{\Phi} ) 
\end{align*}

\remark{The notes use guarded recursion to define baglist. I ended using the index $n$. Need to try the guarded recursion approach.}

The specs for the different methods are:
\begin{itemize}

\item $\hot{T}{\texttt{newbag } ()}{v. \gamma \isBag{v}{\Phi}{\gamma}}$

\item $\hot{\Phi v * \isBag{b}{\Phi}{\gamma}}{\texttt{insert } b \text{ } v}{T}$

\item $\hot{\isBag{b}{\Phi}{\gamma}}{\texttt{remove} b}{v. v = \nonelit \lor \exists x. v = \somelit{x} * \Phi x}$

\end{itemize}

Notice that \texttt{isBag} is also persistent/duplicable.

\remark{Need to implement concurrent client for bags}
