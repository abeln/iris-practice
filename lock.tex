\section{Locks and Coarse-Grained Bags}
\label{section:locks-and-coarse-graned-bags}

\remark{{\footnotesize Coq code: \url{https://github.com/abeln/iris-practice/blob/master/lock.v}}}

This is the spin lock example from Section 7.6 of the notes.

The spin lock is a module with three methods:
\begin{itemize}

\item \texttt{newlock} creates a new lock. Locks are represented as a reference to a boolean. If \texttt{false}, the lock is free; if \texttt{true}, then it's taken.

\item \texttt{acquire} uses a CAS cycle to spin until it can acquire the lock.

\item  \texttt{release} just sets the lock to \texttt{false}.

\end{itemize}

The code is as above:

\begin{verbatim}
  Definition newlock : val := \lam: <>, ref #false.
    
  Definition acquire : val :=
    rec: "acquire" "l" :=
      if: CAS "l" #false #true then #() else "acquire" "l".
    
  Definition release : val := \lam: "l", "l" <- #false.
\end{verbatim}

\subsection{Spec}

The involved RA is $\exclra{\unittyp}$. Notice that unlike other RA constructions, $\exclra{S}$ just requires $S$ to be a \emph{set}, as opposed to another RA.

The exclusive RA is defined by adding an element $\bot$ to the carrier set. For any two elements $x, y \ne \bot$, we have $\prodelem{x}{y} = \bot$. Every element except for $\bot$ is valid.

What this means is that if we own $\ownr{x}{\gamma}$, no other thread can own another $\ownr{y}{\gamma}$, because their product would be invalid.

\subsection{Client: Coarsed-Grained Bags}