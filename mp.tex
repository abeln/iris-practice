\section{Message-Passing Idiom}
\label{message-passing-idiom}

\remark{{\footnotesize Coq code: \url{https://github.com/abeln/iris-practice/blob/master/weakmem.v}}}

This is the ``message passing'' example from the ``Strong Logic for Weak Memory' paper': \url{https://people.mpi-sws.org/~dreyer/papers/iris-weak/paper.pdf}.


\subsection{Code}

The program is simple: we have two variables $x$ and $y$ that start as $0$, and are then mutated by two threads. The second thread loops until $y$ is non-zero, which restricts the order of execution:

\begin{verbatim}
  (* First we have a function `repeat l`, which reads l until its value is non-zero,
     at which point it returns l's value. *)
  Definition repeat_prog : val :=
    rec: "repeat" "l" :=
      let: "vl" := !"l" in
      if: "vl" = #0 then ("repeat" "l") else "vl".

  (* Then we have the code for the example. *)
  Definition mp : val :=
    \lambda: <>,
       let: "x" := ref #0 in
       let: "y" := ref #0 in
       let: "res" := (("x" <- #37;; "y" <- #1) ||| (repeat_prog "y";; !"x")) in
       Snd "res".
\end{verbatim}

\subsection{Specification}

This example uses impredicative invariants: specifically one invariant for $x$ that's embedded inside another invariant that talks about $y$ (notice we're interested in the value of $x$ at the end):
\begin{itemize}
\item The (inner) invariant for $x$ is $Inv_x \triangleq \inv{$\pointsto{x}{37} \lor \ownr{()}{\gamma}$}{l_x}$.

\item The (outer) invariant for $y$ is $Inv_y \triangleq \inv{$\pointsto{y}{0} \lor \pointsto{y}{1} \land  Inv_x$}{l_y}$
\end{itemize}

How should we read the invariants above?
\begin{itemize}
\item 
\end{itemize}

Notice we use the $\exclra{\unittyp}$ RA.