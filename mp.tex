\section{Message-Passing Idiom}
\label{message-passing-idiom}

\remark{{\footnotesize Coq code: \url{https://github.com/abeln/iris-practice/blob/master/weakmem.v}}}

This is the ``message passing'' example from the ``Strong Logic for Weak Memory' paper': \url{https://people.mpi-sws.org/~dreyer/papers/iris-weak/paper.pdf}.


\subsection{Code}

The program is simple: we have two variables $x$ and $y$ that start as $0$, and are then mutated by two threads. The second thread loops until $y$ is non-zero, which restricts the order of execution:

\begin{verbatim}
  (* First we have a function `repeat l`, which reads l until its value is non-zero,
     at which point it returns l's value. *)
  Definition repeat_prog : val :=
    rec: "repeat" "l" :=
      let: "vl" := !"l" in
      if: "vl" = #0 then ("repeat" "l") else "vl".

  (* Then we have the code for the example. *)
  Definition mp : val :=
    \lambda: <>,
       let: "x" := ref #0 in
       let: "y" := ref #0 in
       let: "res" := (("x" <- #37;; "y" <- #1) ||| (repeat_prog "y";; !"x")) in
       Snd "res".
\end{verbatim}

\subsection{Specification}

This example uses impredicative invariants\footnote{Leon points out that we don't really need impredicativity for this example: we might as well have inlined  the inner invariant.}: specifically one invariant for $x$ that's embedded inside another invariant that talks about $y$ (notice we're interested in the value of $x$ at the end):
\begin{itemize}
\item The (inner) invariant for $x$ is $Inv_x \triangleq \inv{$\pointsto{x}{37} \lor \ownr{()}{\gamma}$}{l_x}$.

\item The (outer) invariant for $y$ is $Inv_y \triangleq \inv{$\pointsto{y}{0} \lor \pointsto{y}{1} \land  Inv_x$}{l_y}$
\end{itemize}

Notice we use the $\exclra{\unittyp}$ RA.


How does the verification work?
\begin{itemize}
\item We'll give the left thread  the resources $Inv_y * \pointsto{x}{0}$. The first update of $x$ can be done without problem. The second update for $y$ forces us to ``choose'' the second term in the disjunction. We have to give up ownership of $x$ to the invariant.

\item The right thread gets the resources $Inv_y * \ownr{()}{\gamma}$. The proof is by Lob induction (since the function is recursive). There are two cases to consider, induced by the test of $y$'s value in \texttt{repeat}.

\begin{itemize}
\item If $y = 0$, then we use the induction hypothesis.

\item If $y = 1$, then we know we're in the second case of the invariant and, further, that $\pointsto{x}{37}$ (because the resource is exclusive). In this case, we're able to ``swap'' $\pointsto{x}{37}$ for $\ownr{()}{\gamma}$ when we close the invariant, leaving us with ownership of $\pointsto{x}{37}$ (although this is not required to verify the example).
\end{itemize}

\end{itemize}

\subsection{Getting Ownership of both $x$ and $y$}

I also did a variant of the exercise (suggested by Leon) where we want to end up with ownership of \emph{both} variables.

I ended doing this in quite a ``mechanistic'' way, which suggests possibilities for automation (this had also been suggested by Leon).

The idea is to define an invariant that lists all the possible ``actual'' values of $x$ and $y$:

\[ Inv \triangleq S_1 \lor S_2 \lor S_3 \lor S_4 \]

Each $S_i$ contains the state of the heap plus a ``key'' that's needed to ``access'' the state.

\begin{align*}
S_1 &= \pointsto{y}{0} * \pointsto{x}{0} * \ownr{()}{\gamma_2} *  \ownr{()}{\gamma_3} *  \ownr{()}{\gamma_4} \\
S_2 &= \pointsto{y}{0} * \pointsto{x}{37} * \ownr{()}{\gamma_1} *  \ownr{()}{\gamma_3} *  \ownr{()}{\gamma_4} \\
S_3 &= \pointsto{y}{1} * \pointsto{x}{37} * \ownr{()}{\gamma_1} *  \ownr{()}{\gamma_2} *  \ownr{()}{\gamma_4} \\
S_4  &= \ownr{()}{t_1} * \ownr{()}{t_2} * \ownr{()}{\gamma_1} *  \ownr{()}{\gamma_2} *  \ownr{()}{\gamma_3} \\
\end{align*}

\remark{explain in more detail}

\subsection{\remark{Questions}}
\begin{itemize}
\item Can we open invariants at ``dummy atomic statements''? We might want to do this to ``exchange'' some tokens by others.
\end{itemize}