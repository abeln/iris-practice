\section{Modular Specifications}
\label{section:modular-specs}

\begin{lemma}
$\apointsto{\gamma}{n}{q} * \apointsto{\gamma}{m}{q} \vdash n = m$
\end{lemma}

\begin{proof}
We have $\apointsto{\gamma}{n}{q} = \ownr{(q, n)}{\gamma}$ and $\apointsto{\gamma}{m}{q} = \ownr{(q, m)}{\gamma}$. Then
\begin{align*}
\ownr{(q, n)}{\gamma} * \ownr{(q, m)}{\gamma} & \vdash \ownr{\prodelem{(q, n)}{(q, m)}}{\gamma} & \text{Own-op}\\
& \vdash \prodelem{(q, n)}{(q, m)} \in V & \text{Own-valid} \\
& \vdash (2q, \prodelem{n}{m}) \in V &\\
& \vdash \prodelem{n}{m} \in V & \\
& \vdash n = m 
\end{align*}
\end{proof}

\begin{lemma}
$\apointsto{\gamma}{m}{p} * \apointsto{\gamma}{m}{q} \dashv \vdash \apointsto{\gamma}{m}{p + q}$
\end{lemma}
\begin{proof}
Follows directly from the properties of the resource algebra $\fracra \times \agra{\natra}$.
\end{proof}

\begin{lemma}
$\apointsto{\gamma}{m}{1} \vdash (\upmod{\apointsto{\gamma}{n}{1}})$
\end{lemma}

\begin{proof}
We use \textsc{Ghost-update} and then have to show $\fpupd{(1, m)}{(1,n)}$.  That, suppose that $\prodelem{(1, m)}{(q, m')} \in V$. Then we have to show that $\prodelem{(1, n)}{(q, m')} \in V$.

Notice that $\prodelem{(1, m)}{(q, m')} \in V$ implies $\prodelem{1}{q} \in V$. In turn, this implies $1 + q \in V$, which means $1 + q \le 1$. But this can't be, because we know $q > 0$. This means that there's no such $(q, m')$ and the result follows immediately.
\end{proof}

\subsection{Modular Counter Spec}

The modular counter spec is 

\begin{align*}
\exists Cnt : & Val \rightarrow GhostName \rightarrow InvName \rightarrow Prop.\\
& \pemod{ ( \forall v, \gamma, c. \Cnt{v}{\gamma}{c} \implies \pemod{ \Cnt{v}{\gamma}{c} } ) } \\
\\
& \land \hot{True}{ newCounter(())}{v. \exists \gamma, c. \Cnt{v}{\gamma}{c} * \apointsto{\gamma}{0}{\frac{1}{2}}}_{\varepsilon} \\
\\
&  \land \forall v, \gamma, c, P, Q. (\forall m. \apointsto{\gamma}{m}{\frac{1}{2}} * P \Rrightarrow_{\varepsilon \backslash \{c\}} \apointsto{\gamma}{m}{\frac{1}{2}} * Q(m) ) \implies \\
& \hot{\Cnt{v}{\gamma}{c} * P}{read(v)}{w. \Cnt{v}{\gamma}{c} * Q(w)}_\varepsilon \\
\\
& \land \forall v, \gamma, c P, Q. (\forall m. \apointsto{\gamma}{m}{\frac{1}{2}} * P \Rrightarrow_{\varepsilon \backslash \{c\}} \apointsto{\gamma}{m + 1}{\frac{1}{2}} * Q(m)) \implies \\
& \hot{\Cnt{v}{\gamma}{c} * P}{incr(v)}{w. \Cnt{v}{\gamma}{c} * Q(w)}_{\varepsilon} \\
\\
& \land \forall v, \gamma, c, m, q.  (\apointsto{\gamma}{m}{\frac{1}{2}} * P \Rrightarrow_{\varepsilon \backslash \{ c \}} \apointsto{\gamma}{m + 1}{\frac{1}{2}} * \apointsto{\gamma}{m + 1}{q} * Q) \implies \\
& \hot{\Cnt{v}{\gamma}{c} * \apointsto{\gamma}{m}{q}  * P}{wk\_incr(v)}{w. w = () * \Cnt{v}{\gamma}{c} * \apointsto{\gamma}{m + 1}{q} * Q}_{\varepsilon}
\end{align*}

\subsubsection{Realizing the Spec}

We instantiate the abstract predicate by

\[
\Cnt{v}{\gamma}{c} = \invm{\exists m. \pointsto{v}{m} * \apointsto{\gamma}{m}{\frac{1}{2}}}{c}
\] 

We have four proof obligations to realize the spec:
\begin{enumerate}

\item $Cnt$ is persistent: $\Cnt{v}{\gamma}{c} \vdash \pemod{\Cnt{v}{\gamma}{c}}$

\begin{proof}
Follows by $\textsc{Inv-persistent}$.
\end{proof}

\item $ \vdash \hot{True}{ newCounter(())}{v. \exists \gamma, c. \Cnt{v}{\gamma}{c} * \apointsto{\gamma}{0}{\frac{1}{2}}}$

\begin{proof}
$newCounter$ just allocates a location pointing to $0$, so after the allocation we know $\pointsto{v}{0}$.  We then show the view shift

\[
\vdash \vshift{\pointsto{v}{0}}{\exists \gamma. \pointsto{v}{0} * \apointsto{\gamma}{0}{1}}\
\]

Two rules come in handy for the above:

\infrule
  {\vdash P \implies \upmod{Q}}
  {\vdash \vshift{P}{Q}}

and $P * \upmod{Q}  \vdash \upmod{(P * Q)}$

We can then break up the abstract ownership to get

\[
\vdash \vshift{\pointsto{v}{0}}{\exists \gamma. (\pointsto{v}{0} * \apointsto{\gamma}{0}{\frac{1}{2}}) * \apointsto{\gamma}{0}{\frac{1}{2}}}\
\]

We can then allocate the invariant. \textcolor{red}{I'm not sure which rule to use to allocate the invariant at this point.}

\end{proof}

\item  (\textbf{read})

\begin{align*}
(\forall m. \apointsto{\gamma}{m}{\frac{1}{2}} * P \Rrightarrow \apointsto{\gamma}{m}{\frac{1}{2}} * Q(m) ) \implies\\
 \hot{\Cnt{v}{\gamma}{c} * P}{read(v)}{w. \Cnt{v}{\gamma}{c} * Q(w)}
\end{align*}

We get to assume $\forall m. \apointsto{\gamma}{m}{\frac{1}{2}} * P \Rrightarrow \apointsto{\gamma}{m}{\frac{1}{2}} * Q(m) $ and need to show
\[ \hot{\Cnt{v}{\gamma}{c} * P}{read(v)}{w. \Cnt{v}{\gamma}{c} * Q(w)} \]

\begin{itemize}
\item We know that $read(v)$ is just $!v$.
\item To read $v$, we open the invariant, and get $\latermod{\exists m. \pointsto{v}{m} * \apointsto{\gamma}{m}{\frac{1}{2}}}$.
\item After reading $v$, we know that the result is some $m$ and also know that $\pointsto{v}{m} * \apointsto{\gamma}{m}{\frac{1}{2}}$. We also know $P$ from the hypothesis.
\item We can then instantiate our view shift hypothesis to conclude $\apointsto{\gamma}{m}{\frac{1}{2}} * Q(m)$.
\item We can close the invariant because the physical and abstract state of the counter agree.
\end{itemize}

\item (\textbf{incr})

\begin{align*}
& \forall v, \gamma, c P, Q. (\forall m. \apointsto{\gamma}{m}{\frac{1}{2}} * P \Rrightarrow_{\varepsilon \backslash \{c\}} \apointsto{\gamma}{m + 1}{\frac{1}{2}} * Q(v)) \implies \\
& \hot{\Cnt{v}{\gamma}{c} * P}{incr(v)}{w. \Cnt{v}{\gamma}{c} * Q(w)}_{\varepsilon} 
\end{align*}

We have 
\begin{verbatim}
incr(v) =
  let m = !v in
  let m' = m + 1 in
  if (CAS(v, m, m')) then n else incr(v)
\end{verbatim}

We assume the view shift and then need to show
\[ \hot{\Cnt{v}{\gamma}{c} * P}{incr(v)}{w. \Cnt{v}{\gamma}{c} * Q(w)}_{\varepsilon}  \]

\begin{itemize}
\item We proceed by Lob induction, since incr is recursive.

\item To read $!v$, we open the invariant, do the load, and close the invariant. The result is that $m = n$, for some $n$.

\item Then we set $m' = n + 1$.

\item Now we're at the CAS, and we know $\Cnt{v}{\gamma}{c} * P$. Again we open the invariant, and then we know $\exists n'. \pointsto{v}{n'} * \apointsto{\gamma}{n'}{\frac{1}{2}} * P$. We consider two cases: $n' = n$ or $n' \ne n$.

\begin{itemize}
\item If $n' \ne n$, then we recurse and can use the induction hypothesis.

\item If $n' = n$, then we know $\pointsto{v}{n} * \apointsto{\gamma}{n}{\frac{1}{2}} * P$.
At that point, we can use the view shift to conclude $\apointsto{\gamma}{n + 1}{\frac{1}{2}} * Q(n)$.  Because the CAS succeeds, we also know that $\pointsto{v}{n + 1}$, so we can close the invariant.

\end{itemize}

\end{itemize}

\item (\textbf{wk\_incr})

\begin{align*}
\forall v, \gamma, c, m, q.  (\apointsto{\gamma}{m}{\frac{1}{2}} * P \Rrightarrow_{\varepsilon \backslash \{ c \}} \apointsto{\gamma}{m + 1}{\frac{1}{2}} * \apointsto{\gamma}{m + 1}{q} * Q) \implies \\
\hot{\Cnt{v}{\gamma}{c} * \apointsto{\gamma}{m}{q}  * P}{wk\_incr(v)}{w. w = () * \Cnt{v}{\gamma}{c} * \apointsto{\gamma}{m + 1}{q} * Q}_{\varepsilon}
\end{align*}

We have 
\[
wk\_incr(v) \triangleq v \gets 1 + !v
\]

We assume 
\[
\apointsto{\gamma}{m}{\frac{1}{2}} * P \Rrightarrow_{\varepsilon \backslash \{ c \}} \apointsto{\gamma}{m + 1}{\frac{1}{2}} * \apointsto{\gamma}{m + 1}{q} * Q
\]

and need to show
\[
\hot{\Cnt{v}{\gamma}{c} * \apointsto{\gamma}{m}{q}  * P}{wk\_incr(v)}{w. w = () * \Cnt{v}{\gamma}{c} * \apointsto{\gamma}{m + 1}{q} * Q}_{\varepsilon}
\]

\begin{itemize}

\item We use the bind rule to focus on $!v$, and open the invariant. We get to assume
\[
\latermod{(\exists m'. \pointsto{\gamma}{m'} * \apointsto{\gamma}{m'}{\frac{1}{2}})} * \apointsto{\gamma}{m}{q} * P
\]

\item Because of the agreement construction and the lemma we proved above, we'll get $m' = m$.

\item After reading $!v$, we have $\Cnt{v}{\gamma}{c} * \apointsto{\gamma}{m}{q} * P$.

\item Then we need to do the write $v \gets 1 + m$. Again we open the invariant, and again we can know the value of the pointer ($m$). We have $\apointsto{\gamma}{m}{\frac{1}{2}} * \apointsto{\gamma}{m}{q} * P * \pointsto{v}{m}$.

\item After the write, we know
\[
 \pointsto{v}{m + 1} * \apointsto{\gamma}{m}{\frac{1}{2}} * \apointsto{\gamma}{m}{q} * P
\]

\item We now use our assumption about the view shift to conclude 
\[
\apointsto{\gamma}{m + 1}{\frac{1}{2}} * \apointsto{\gamma}{m + 1}{q} * Q
\]

So in total, we own
\[
 \pointsto{v}{m + 1}  * \apointsto{\gamma}{m + 1}{\frac{1}{2}} * \apointsto{\gamma}{m + 1}{q} * Q
\]

\item We can close the invariant to get
\[
\Cnt{v}{\gamma}{c} * \apointsto{\gamma}{m + 1}{q} * Q
\]

as needed.

\item \textcolor{red}{We didn't use the restriction that the view shift happens with the mask $\varepsilon \backslash \{ c \}$.}

\item \textcolor{red}{The lecture notes are missing $\apointsto{\gamma}{m + 1}{q}$ in the postcondition.}

\item \textcolor{red}{What if we didn't provide the $\apointsto{\gamma}{m}{q}$ resource?}  In that case, the Hoare triple would be
\[
\hot{\Cnt{v}{\gamma}{c} * P}{wk\_incr(v)}{w. w = () * \Cnt{v}{\gamma}{c} * Q}_{\varepsilon}
\]

We wouldn't know the value of the counter when $wk\_incr(v)$ finishes. This makes sense, because other threads could be incrementing the counter at the same time.

\end{itemize}

\end{enumerate}

\subsubsection{Derived Sequential Spec}

This is the desired sequential spec:

\begin{align*}
\exists SeqCnt : & Val \rightarrow GhostName \rightarrow InvName \rightarrow \mathbb{N} \rightarrow Prop.\\
& \land \hot{True}{ newCounter(())}{v. \exists \gamma, c. \SCnt{v}{\gamma}{c}{0}} \\
\\
& \land \hot{\SCnt{v}{\gamma}{c}{n}}{read(v)}{w. w = n * \SCnt{v}{\gamma}{c}{n}} \\
\\
& \land \hot{\SCnt{v}{\gamma}{c}{n}}{incr(v)}{w. w = n * \SCnt{v}{\gamma}{c}{n + 1}} \\
\end{align*}

Where we set

\[
\SCnt{v}{\gamma}{c}{n} \triangleq \Cnt{v}{\gamma}{c} * \apointsto{\gamma}{n}{\frac{1}{2}}
\]

\textbf{Proofs}:
\begin{itemize}
\item (\textbf{newCounter}):  We know

\[
\hot{True}{ newCounter(())}{v. \exists \gamma, c. \Cnt{v}{\gamma}{c} * \apointsto{\gamma}{0}{\frac{1}{2}}}_{\varepsilon}
\]

and need to derive
\[
\hot{True}{ newCounter(())}{v. \exists \gamma, c. \SCnt{v}{\gamma}{c}{0}}
\]

This follows immediately from the definition of $SeqCnt$.

\item (\textbf{read}): We know

\begin{align*}
(\forall m. \apointsto{\gamma}{m}{\frac{1}{2}} * P \Rrightarrow \apointsto{\gamma}{m}{\frac{1}{2}} * Q(m) ) \implies\\
 \hot{\Cnt{v}{\gamma}{c} * P}{read(v)}{w. \Cnt{v}{\gamma}{c} * Q(w)}
\end{align*}

and need to derive

\[
\hot{\SCnt{v}{\gamma}{c}{n}}{read(v)}{w. w = n * \SCnt{v}{\gamma}{c}{n}}
\]

We set
\begin{itemize}
\item $P = \apointsto{\gamma}{n}{\frac{1}{2}}$. This means that

\begin{align*}
\SCnt{v}{\gamma}{c}{n} &= \Cnt{v}{\gamma}{c} * \apointsto{\gamma}{n}{\frac{1}{2}}\\
& = \Cnt{v}{\gamma}{c}  * P
\end{align*}

\item $Q = \lambda w. (w = n * \apointsto{\gamma}{n}{\frac{1}{2}})$. \textcolor{red}{What we're doing here is instantiating $P$ and $Q$ to give us the resources we need in the pre and post conditions.}

\item All that remains is to show
\[
\vdash \forall m. \apointsto{\gamma}{m}{\frac{1}{2}} * \apointsto{\gamma}{n}{\frac{1}{2}} \Rrightarrow \apointsto{\gamma}{m}{\frac{1}{2}} * (m = n * \apointsto{\gamma}{n}{\frac{1}{2}})
\]

We use this rule to drop the persistent modality at the top level (obscured by the view shift):

\infrule
  {\vdash P \implies \upmod{Q}}
  {\vdash \vshift{P}{Q}}
  
Then we have to show

\[
\apointsto{\gamma}{m}{\frac{1}{2}} * \apointsto{\gamma}{n}{\frac{1}{2}} \implies \upmod{ \apointsto{\gamma}{m}{\frac{1}{2}} * (m = n * \apointsto{\gamma}{n}{\frac{1}{2}})}
\]

The agreement construction gives us $m = n$, and then the result follows from $P \vdash \upmod{P}$.

\textcolor{red}{The $m = n$ part was pretty magical. It is important because it links the physical and logical states. In this case, it worked because we're using the agreement construction. Do we always use the agreement construction for these modular specs?}

%\textcolor{red}{It's pretty crazy we can say something about the physical state by refining an abstract specification that only mentions the return value!}

\end{itemize}

\item (\textbf{incr}):

We know

\begin{align*}
& \forall v, \gamma, c, P, Q. (\forall m. \apointsto{\gamma}{m}{\frac{1}{2}} * P \Rrightarrow_{\varepsilon \backslash \{c\}} \apointsto{\gamma}{m + 1}{\frac{1}{2}} * Q(v)) \implies \\
& \hot{\Cnt{v}{\gamma}{c} * P}{incr(v)}{w. \Cnt{v}{\gamma}{c} * Q(w)}_{\varepsilon} 
\end{align*}

And need to show

\begin{align*}
\hot{\SCnt{v}{\gamma}{c}{n}}{incr(v)}{w. w = n * \SCnt{v}{\gamma}{c}{n + 1}}
\end{align*}

We set
\begin{itemize}

\item $P = \apointsto{\gamma}{n}{\frac{1}{2}}$

\item $Q = \lambda x. x = n * \apointsto{\gamma}{n + 1}{\frac{1}{2}}$ \textcolor{red}{The notes are missing the $x = n$ part.}

\end{itemize}

The pre and post conditions match up, so we're left with showing

\[
\forall m. \apointsto{\gamma}{m}{\frac{1}{2}} * \apointsto{\gamma}{n}{\frac{1}{2}} \Rrightarrow_{\varepsilon \backslash \{c\}} \apointsto{\gamma}{m + 1}{\frac{1}{2}} * m = n * \apointsto{\gamma}{n + 1}{\frac{1}{2}}
\]

\begin{itemize}
\item Again, from the agreement construction, we know that $m = n$, so we get

\[
\apointsto{\gamma}{n}{\frac{1}{2}} * \apointsto{\gamma}{n}{\frac{1}{2}} \implies \upmod{ (\apointsto{\gamma}{n + 1}{\frac{1}{2}} * \apointsto{\gamma}{n + 1}{\frac{1}{2}}) }
\]
\end{itemize}

Since we have full ownership of $\gamma$, we can perform the frame-preserving update.

\end{itemize}

\subsubsection{Derived Concurrent Spec}