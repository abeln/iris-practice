\section{Bags with Helping}
\label{section:bags-with-helping}

\remark{{\footnotesize Coq code: \url{https://github.com/abeln/iris-practice/blob/master/helping.v}}}

This is a larger example, also from the lecture notes. The idea of this example is to implement a stack with an optimization that consists of a ``mailbox''. This mailbox is just a memory location where threads can temporarily please and remove items so as to avoid having to push and pop in certain cases.

Specifically, the mailbox comes in handy in the following situation:
\begin{itemize}
\item Suppose we have two threads $A$ and $B$, and the stack is currently empty.
\item Thread $A$ wants to push an element to the stack.
\item Thread $B$ wants to remove an element.
\item Without mailboxes, $B$ would wait until $A$ pushes, then $B$ would pop.
\item With the mailbox, $A$ instead of pushing places its element in the mailbox. $B$ notices it and then removes the element. No pushs or pops were necessary. 
\item \remark{Question: how much time/instructions are we actually saving with this?}
\end{itemize}

\subsection{Offers}

The mailbox is itself implemented in terms of a lower-level abstraction: an ``offer''.

An offer is a pair $(v, l)$, where $v$ is the value in the offer and $l$ is a location indicating the offer status:
\begin{itemize}
\item $\pointsto{l}{0}$ is the initial state of the offer
\item $\pointsto{l}{1}$ means the offer has been accepted
\item $\pointsto{l}{2}$ means the offer has been revoked (an offer can only be revoked by the thread that created it)
\end{itemize}

\textbf{Offer Code}

\begin{verbatim}
  Definition mk_offer : val := \lam: "v", ("v", ref #0).

  Definition revoke_offer : val :=
    \lam: "off",
      let: "v" := Fst "off" in
      let: "l" := Snd "off" in
      if: (CAS "l" #0 #2) then SOME "v" else NONE.

  Definition accept_offer : val :=
    \lam: "off",
      let: "v" := Fst "off" in
      let: "l" := Snd "off" in
      if: (CAS "l" #0 #1) then SOME "v" else NONE.
\end{verbatim}

\textbf{Offer Specification}

The representation predicate for offers uses the state transition ``trick'' to encode the three states of the offer, plus the fact that only the creator can revoke an offer.

\[
\isoffer{o}{\gamma} \triangleq \exists v, l. o = (v, l) * \invm{\pointsto{l}{0} * \Phi v \lor \pointsto{l}{1} \lor (\pointsto{l}{2}) * \ownr{()}{\gamma}}{l}
\]

\begin{itemize}
\item Notice this is duplicable, as usual.
\item Notice the predicate is indexed by $\gamma$, so an offer can only be revoked by a thread holding $\ownr{()}{\gamma}$, the ``key''.
\item If $\pointsto{l}{0}$, we also know $\Phi v$. This ensures that every element in the bag satisfies the requisite predicate $\Phi$. This is as per the bag spec in a previous example.
\end{itemize}

The method specs are
\begin{itemize}

\item $\hot{\Phi v}{\texttt{mk\_offer v}}{o. \exists \gamma. \isoffer{o}{\gamma}}$

\item $\hot{\isoffer{o}{\gamma}}{\texttt{accept\_offer o}}{v. v = None \lor \exists w. v = Some(w) * \Phi w }$

\item $\hot{\isoffer{o}{\gamma} * \ownr{()}{\gamma}}{\texttt{revoke\_offer o}}{v. v = None \lor \exists w. v = Some(w) * \Phi w }$. \remark{The specs of accept and revoke are identical, except we require ownership of the key for revoke.}
\end{itemize}

\subsection{Mailboxes}

Offers are values, so they can't be mutated (well, the location component of an offer, which indicates state, can be mutated).  A mailbox is then a location pointing to an offer.

Following the lecture notes, we implement mailboxes as a location, plus two closures over that location:

\begin{verbatim}
  Definition mk_mailbox : val :=
    \lam: <>,
       let: "r" := ref NONEV in

       let: "put" :=
         (\lam: "v",
           let: "o" := mk_offer "v" in
           "r" <- SOME "o";;
           revoke_offer "o")
       in

       let: "get" :=
         (\lam: "v",
           match: !"r" with
             NONE => NONEV
           | SOME "o" => accept_offer "o"
           end)
       in
        
       ("put", "get").
\end{verbatim}

\textbf{Comments:}
\begin{itemize}
\item \remark{What are the pros/cons of this style with closures preferred over just declaring each function at the top level?}

\item It's interesting that \texttt{put} creates an offer and ``right away'' revokes it. Of course, since \texttt{put} and \texttt{get} are not synchronized, there's room for a concurrent execution of \texttt{get} to remove the item in the offer (and the notes do mention that in a realistic setting we'd want a timeout here).

\item \remark{Is there a version of separation logic for reasoning about time? e.g. reasoning about programs with real-time constraints? Would we run into liveness issues?}

\item See ``separation logic with time credits'' for doing amortized complexity analysis: \url{https://link.springer.com/article/10.1007/s10817-017-9431-7}.
\end{itemize}

The representation predicate for mailboxes is straightforward:

\[
\ismailbox{l} \triangleq = \pointsto{l}{None} \lor \exists o, \gamma. \pointsto{l}{Some(o)} * \isoffer{o}{\gamma}
\]

Within the proof of the spec (I haven't shown the spec yet), we need to box the above in an invariant.

The spec for \texttt{mk\_mailbox} is what we would expect (both \texttt{get} and \texttt{put} return either a $None$ or $Some(v)$, for which we know $\Phi v$). \remark{The one interesting thing is this spec uses nested Hoare triples.}

\subsection{Stack}

Finally we get to the stack implementation and its bag specification. Again we use the same closure style. See Figure~\ref{figure:stack-code}.

\begin{figure}
\begin{verbatim}
 Definition mk_stack : val :=
    \lam: <>,
       let: "mb" := mk_mailbox #() in
       let: "put" := Fst "mb" in
       let: "get" := Snd "mb" in

       (* The stack is a pointer p that points to either
          - NONEV, if the stack is empty
          - SOMEV l, if the stack is non-empty.
            l in turn points to a pair (h, t), where h is the top of the stack
            and `t` is the rest of the stack. *)
       
       let: "stack" := ref NONEV in

       (* Push an element into the stack. Returns unit. *)
       let: "push" :=
          rec: "push" "v" :=
            match: ("put" "v") with
              NONE => #()
            | SOME "v" =>
              let: "curr" := !"stack" in
              let: "nstack" := SOME (ref ("v", "curr")) in
              if: (CAS "stack" "curr" "nstack") then #() else ("push" "v")
            end
       in

       (* Pop an element from the stack. If the stack is empty, return None,
          otherwise return Some head. *)
       let: "pop" :=
        rec: "pop" <> :=
          match: ("get" #()) with
            SOME "v" => SOME "v"
          | NONE =>
            match: !"stack" with
              NONE => NONEV
            | SOME "l" =>
              let: "p" := !"l" in
              let: "head" := Fst "p" in
              let: "tail" := Snd "p" in
              if: (CAS "stack" (SOME "l") "tail") then SOME "head" else ("pop" #())
            end
          end
       in

       ("push", "pop").
\end{verbatim}
\caption{Stack implementation with mailboxes}
\label{figure:stack-code}
\end{figure}

A lot of the complexity in the specification comes from the fact that, unlike in the lecture notes, the Coq mechanization of Iris cannot do a \texttt{CAS} on arbitrary values, only on \emph{unboxed} ones. In particular, we can compare locations, but not \emph{pairs}, with a \texttt{CAS}. 

The specifications of push and pop are the regular bag specifications we've already seen.

The tricky thing is coming up with the invariant for the stack. I stole the idea from the solution in the Iris repo. However, the solution in the repo uses guarded recursion, which I didn't use, so I had to adapt things.

First, the stack invariant:
\[
\stackinv{l} \triangleq \exists o, ls. \pointsto{l}{\texttt{oloc\_to\_value o} * \isstack{o}{ls} }
\]

\begin{itemize}

\item $o$ here is an \emph{Option} that wraps a \emph{location}. What we really want to say is that either $l$ points to None or to $Some(l')$, but if we say it like that then we introduce duplication in the proofs. \remark{This factoring of useful information in invariants seems important.}

\item The \texttt{oloc\_to\_value} helper is defined thus

\begin{verbatim}
  Definition oloc_to_val (o : option loc) : val :=
    match o with
      None => NONEV
    | Some l => SOMEV #l
    end.
\end{verbatim}

\item The representation predicate for stacks is generated by the function

\begin{verbatim}
  Fixpoint is_stack (o : option loc) (ls : list val) : iProp :=
    match ls with
      nil => o = None
    | x :: xs => \exists (l : loc) (o' : option loc),
      o = Some l * l ->{-}(x, oloc_to_val o') * \Phi x * (is_stack o' xs)
  end.
\end{verbatim}

\remark{TODO: find a way to phrase the predicate without using the option trick}

Notice here we have the fractional points-to predicate. \remark{This means that when we open the invariant we'll get write access to the first element of the list, but only read access to the rest of the elements. This is crazy!}

\end{itemize}